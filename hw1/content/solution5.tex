\section*{Solution - Question 5}
%\label{sec:sol5}
Starting from the general system described in (\ref{eqn:syseqs}), the goal for this question is to generalize the system for more than 2 species.

% Let's consider a first case, in which there are $N=3$ species in the ecosystem. Under this assumption, the expanded system becomes:

% $$
%     \left\{\begin{aligned} 
%       \dot{x}_1(t) = x_1(a_1 + r_{11}x_1 +  r_{12}x_2 +  r_{13}x_3) \\
%       \dot{x}_2(t) = x_2(a_2 + r_{21}x_1 +  r_{22}x_2 +  r_{23}x_3) \\
%       \dot{x}_3(t) = x_3(a_3 + r_{31}x_1 +  r_{32}x_2 +  r_{33}x_3)
%     \end{aligned}\right.
% $$

We introduce a change in notation. Now, for each specie $i$, identified by $x_i$, the constant that shows if eats grass becomes $a_i$, and the relationship between this specie with another specie $j$ is denoted by $r_{ij}$.

If we generalize the system for $N$ species, we obtain for each one (recall $a_i,r_{ij} \in \mathbb{R}, i,j=1,...,N$):

% $$
% 	\dot{x}_i = x_i\Big (a_i + \sum_{j=1}^N r_{ij}x_j\Big) \quad a_j,b_{ji} \in \mathbb{R}, j=1,...,N
% $$
$$ \dot{x}_i(t) = x_i \left(a_i + \sum_{j=1}^N r_{ij}x_j\right)$$
