\section*{Solution - Question 1}
%\label{sec:sol1}

A sketch of the signals $\tin$, $\tout$, $\tind$ and $\toutd$ when the input is the one described in equation (\ref{eq:inputBL}) is presented in figure \ref{fig:sketch}. We analyse the effect of the back-lash by looking at the signals $\tout$ and $\toutd$.

Because of the initial condition $\tout(0)=-\Delta$, in $t=0$ there is contact. In addition, the input starts moving in the positive sense, which translates this movement directly from the input to the output. 

Once the input switches its sense, we have to wait $2\Delta$, which is the time employed to move from the contact position to the other one when the velocity is 1. Furthermore, if $2\Delta > 1$, there will not be any movement, since the contact will not be reached before the movement stops. 

Considering that this condition is not fulfilled, and thus there is movement in this part, the final position will be $1-\Delta - (1-2\Delta) = \Delta$, which corresponds to the initial condition minus the time that there is movement.