\section*{Solution - Question 4}
%\label{sec:sol4}
If BIBO stability could not be ensured, one of the possible causes is that the new linear system is not stable. 

As it is stated in ~\cite[p. 99]{Oppenheim:2002:Signals-and-systems}, a linear and time-invariant system is stable if the impulse response is absolutely integrable:

$$
\int_{-\infty}^{\infty} |h(\tau)| d\tau = C < \infty
$$

Since we are given the Laplace-domain system, we have to compute the inverse transform:

$$
g(t) = \mathcal{L}^{-1} \{G(s)\} = 
$$
$$
= \mathcal{L}^{-1} \left\{ \frac{K}{s(1+sT)} \right\} = K \left( 1 - e^{-t/T} \right) u(t)
$$

Where $u(t)$ denotes the Heaviside step function. Now we apply the theorem:

$$
\int_{-\infty}^{\infty} |g(\tau)| d\tau = \int_{0}^{\infty} \left| K \left( 1 - e^{-t/T} \right) \right| d\tau =
$$
$$
= \int_{0}^{\infty} K d\tau + \int_{0}^{\infty} Ke^{-t/T}  d\tau = K( T + \lim_{\tau \to \infty}\tau) \to \infty 
$$

So, the system $G(s)$ is not stable. This implies, for the Small Gain Theorem and Passivity Theorem stated in question \textbf{3}:

\begin{enumerate}
\item The Small Gain Theorem requires that both systems must be stable. Since one of them is not stable, we cannot conclude BIBO stability using this theorem. 

\item For the Passivity Theorem, we required that $G(s)$ must be strictly passive. Nevertheless, we can only apply the definition of passivity if and only if the system is stable. Since this requirement is not fulfilled, $G(s)$ cannot be passive nor strictly passive. Under these circumstances, Passivity theorem does not give us any conclusion. 
\end{enumerate}

Now we are going to analyse what happens if we consider the Circle Criterion. As it was showed in question \textbf{3}, the requisite imposed to $G(s)$ is to have no poles in the right half plane. Considering $T=1$, the system in equation (\ref{eq:gs}) turns out to be:

$$ G = \frac{K}{s(1+s)}$$

Whose poles are placed in $s = \{0,-1 \}$. In addition, the other condition that leads to BIBO stability is to not intersect the line Re $= -1$. This condition, for $G(s)$, is equivalent to making sure that:
$$
 \textrm{Re }G(j\omega) =\textrm{Re } \frac{K}{-\omega^2+j\omega}= -\frac{K\omega^2}{\omega^4+\omega^2} > -1$$
For $\omega = 0$ the function is not defined, thus first we check the condition  $\forall \omega \neq 0$, then we check the condition for $\omega \to 0$:
\begin{align*}
\frac{K}{\omega^2+1} < 1 \quad \forall \omega \neq 0\\
K-1 < \omega^2 \quad \forall \omega \neq 0
\end{align*}
Which holds $\forall \omega \neq 0$ as long as $K-1 \leq 0 \Rightarrow K \in (0,1]$, since $K>0$. Now we check  the limit for $\omega \to 0$:
$$\lim_{\omega \to 0}-\frac{K\omega^2}{\omega^4+\omega^2}= \lim_{\omega \to 0}-\frac{K}{\omega^2+1}  =-K>-1$$
%$$
%\min \textrm{Re }G(j\omega) = \min \textrm{Re }\frac{K}{-\omega^2+j\omega} = -K 
%$$

%This result gives us the maximum value for $K$, which has to be compared to $-\frac{1}{k_2} = -1$. Therefore, the maximum value is $K=1$.

So, the closed-loop system is BIBO stable for $K \in (0, 1)$, according to the Circle Criterion.
