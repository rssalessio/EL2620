\section*{Solution - Question 3}
\label{sec:sol3}
Now we analyse the system when the pilot is in an "emergency situation", and he is replaced by a "relay pilot" that models the fact that the pilot is panicking and he is trying to compensate the error $\theta_{\text{ref}}-\theta$ with maximum command signals. This behaviour may induce sustained oscillations in the system (PIO - Pilot Induced Oscillations). This is usual for system with slow unstable dynamics, such as the aircraft, that cannot respond fast enough to the command input. 

To analyse the PIO mode we can make use of the describing function method ~\cite[p. 280]{Khalil:2002:Nonlinear-systems:vh} to check for a possible periodic solution. Suppose the error $e(t)=\theta_{\text{ref}}-\theta$ is the input to the non linearity and also $e(t)$ is oscillating with amplitude $A$ and period $\omega$. Let $u(t)$ be the output of the non linearity  and suppose the plant is low-pass filter such that $|G(jn\omega)| \ll |G(j\omega)|$ for $n \geq 2$, then we may use the describing function method to  model the non linearity, so that it is replaced by a function $$N(A,\omega)= \frac{b_1(\omega)+ja_1(\omega)}{A}$$ where $b_1,a_1$ are the first Fourier coefficients of the signal $u(t)$. \\
This gives as output of the plant an oscillating signal of maximum amplitude: $$\textrm{max}|y(t)| \approx |G(j\omega)|\sqrt{a_1^2+b_1^2}= |G(j\omega)||N(A,\omega)|A$$ 
The describing function for a relay is:
$$N(A) = \frac{4D}{\pi A}$$
where $A$ is the amplitude of the oscillation and $D$ is the output value, in absolute sense, of the relay system, and in our case $D=0.2$.\\
To check for existence of period solution the idea is that we have sustained oscillations with pulsation $\omega$ if the open loop-gain is $1$ and the  phase-lag is $-\pi$ for that pulsation:
\begin{equation}\label{eq:cond1}
G(j\omega)N(A)=-1 \Leftrightarrow G(j\omega) = -\frac{1}{N(A)}
\end{equation}
Notice that because of that condition, the output in module has maximum amplitude:
$$\textrm{max} |y(t)| \approx  |G(j\omega)||N(A,\omega)|A = A$$ 
In our case the describing function gives no phase contribution, so we first check for which $\omega $:
\begin{equation} \label{eq:cond2}
\textrm{arg}G(j\omega)=-\pi
\end{equation}
To do so we linearised the system and analysed the Bode and Nyquist plots (figures \ref{fig:sim2bode}, \ref{fig:sim2nyquist}).\\
By checking the plots, the condition (\ref{eq:cond2}) is satisfied for $\omega \approx 2.77 \frac{\textrm{rad}}{\textrm{sec}}$: the Nyquist curve for that pulsation has imaginary part which is almost 0 and  real part $-0.402$. In fact $|G(j2.77)| = -7.92 dB \approx 0.402$.\\
Then we need to find for which $A$ we may obtain oscillations, by using the condition (\ref{eq:cond1}):
$$ G(j2.77) \approx -0.402 = -\frac{\pi A}{4D} \Rightarrow A \approx 0.1024$$
The period of the oscillation is given by: $$\frac{2\pi}{\omega} = \frac{2\pi}{2.77}\approx 2.27 \textrm{ seconds} $$ By simulating the system we see the aircraft oscillating around the set point $1 \text{rad}$ of $\pm 0.1 \text{rad}$ with period $\sim 2.34$ seconds, which is almost what we obtained with the describing function analysis. So the prediction, using the describing function analysis, gives very good results for this case.