\clearpage % Because it gives no credits, this section may not be included within the 4 pages. 

\section*{Solution - Extra Question (6)}
\label{sec:sol6}
The model considered so far does not model 	the rate limitations imposed by the hydraulic servos. This is an important aspect because when a rate limiter is saturated it means that the control signal sent to the actuator is increasing/decreasing more quickly than what the actuator is able to output: this leads to a reduction of the phase margin, because the output of the rate limiter has a phase shift compared to the input, and it increases the possibility of PIO.\\This is easily explained by the fact that the phase shift also affects the pilot, thus creating a time delay effect, happening when the pilot starts to command the aircraft with large and rapid inputs, like in the relay model. 
In industry, software rate limiters are applied to the control commands in order to prevents the hydraulic servos from rate limiting ~\cite[p. 3944]{paper3}, and those software rate limiters are used in conjunction with algorithms or filters to provide phase compensation. \\
For an input signal $u(t)$ the rate limiting effect happens if:
\begin{equation}\label{eq:cond} \lim_{\delta \to 0} \Big|\frac{u(t)-y(t-\delta)}{\delta}\Big|>r
\end{equation}
%where $r \text{[unit/s]}$ is the rate limit and $y(t)$ the output of the rate limiter. If $y(t)=u(t)$, rate limiting will happen when $|\dot{u}(t)|>r$. If $y(t) \neq u(t)$ it means that the output has rate limited, i.e. $|\dot{y}(t)|=r$.
 %Thus, we have again $y(t)=u(t)$ if:
%$$ a\sin (\omega t)=  \textrm{sgn}(\ddot{u}(t_0))r(t-t_0)+a\sin(\omega t_0)$$
%$$ \Rightarrow
%a\frac{\sin(\omega t) - \sin(\omega t_0)}{t-t_0}=\textrm{sgn}(\ddot{u}(t_0))r$$
where $r \text{[unit/s]}$ is the rate limit value. When the rate limiting effect starts to apply, we have $|\dot{y}(t)| = |r|$.\\
Condition (\ref{eq:cond}) for a rate limiting effect to happen may be simplified to $\frac{r}{a\omega}<1$ ~\cite[p. 3945]{paper3}, for sinusoidal input $u(t)=a\sin(\omega t), a>0$,  In fact, suppose $y(t)=u(t)$,  then the rate limiting effect will start when condition \ref{eq:cond} is satisfied:
\begin{align*}
&\lim_{\delta \to 0} \Big|\frac{u(t)-y(t-\delta)}{\delta}\Big| = |\dot{u}(t)|  = |a \omega \cos(\omega t)| > r \\
&\Rightarrow \frac{r}{a \omega} < |\cos(\omega t)|  \Rightarrow \frac{r}{a \omega} < 1
\end{align*}
 Therefore the amplitude and pulsation of a signal are very important, they may determine the activation of the rate limiting effect. Notice that if rate limiting started at $t_0$, then: $$y(t) = \textrm{sgn}(\dot{u}(t_0)) r(t-t_0)+u(t_0)$$\\
Rate limiters were introduced as in article ~\cite[p. 3947]{paper3} in our analysis, and some values of $r$ were tested in order to analyse what happens to the limit cycle. The model changed in the following way (figure \ref{fig:sol6circuit}): rate limiters are now present on both the pilot command and on the sum of the pilot command and feedback command, as in ~\cite[p. 3947]{paper3}. The system was simulated using the three designs and using a rate limiting value of $r=0.5$ first (figure \ref{fig:sol6lowlimit}) and $r=1$ later (figure \ref{fig:sol6highlimit}). Obviously, for $r \to \infty$ the system output will converge to the output of the system that has no rate limiters. For $r=0.5$ the amplitude of the oscillations for \emph{design2} and \emph{design3}  increases, this is due to the fact that $K_f>1$ and the gain of the input command is increased, thus activating the rate limiting effect. Instead, the output for \emph{design1} for both $r=0.5$ and $r=1$ does not change: this is easily explained since $K_f<1$, thus lowering the gain of the signal. Moreover the internal feedback loop does not influence so much since the rate of that loop is not very high, thus there is no  rate limitation effect. For $r \to 0$ obviously we have instability.
\\ \\
A simple method to avoid the rate limiting effect is to either  reduce the gain or use an additional low pass filter on the pilot input, in order to affect $u(t)$. Although being a solution, it reduces the aircraft response ~\cite[p. 3944]{paper3}. Other ideas may be to reduce gain as function of frequency or as function of the input rate, use feedforward or logical expressions ~\cite[p. 20]{paper4}. Early methods also used the idea of differentiating, use saturation and integrate: this permits to have $\textrm{sgn}(\dot{u})=\textrm{sgn}(\dot{y})$, thus perfect phase compensation. But after rate limiting $y \neq u$, since  saturation removed some information ~\cite[p. 20]{paper4}. \\ \\
In \cite{paper3} the idea of using a feedback loop to phase compensate the rate limiter is used, and the inspiration came from anti-windup methods. The error $e(t)=y(t)-u(t)$ is fed back into a filter, in order to stabilise the controller. There are three main variants for this method, and they were all tested using as input $u(t)=5\sin(t)$ for $t \leq 10$, $u(t)=0$ for $t>10$. The results can be seen in figure \ref{fig:simfilters}, where the rate limiters have $r=1$.\\
The three main variants are:
\begin{enumerate}
\item \emph{Feedback with integrator}: the error is fed back into an integrator. This provide good phase compensation, especially for low frequency signals, but where rate limiting ceases there may be a bias in the output because of the integration ~\cite[p. 3945]{paper3}. An example of the bias effect can be seen in figure \ref{fig:simfilters} for $t>10$.
\item \emph{Feedback with low pass filter}: the error is fed back into a low pass filter. In this way when rate limiting ceases there is no bias. When rate limiting starts $e(t) < 0$, thus reducing the input signal. Despite not giving any bias, it gives less phase compensation compared to an integrator ~\cite[p. 3945]{paper3}. Another drawback is that for high frequencies the circuit almost blocks.
\item \emph{Feedback with low pass filter and bypass}: the idea is to use the feedback with low pass filter circuit but using only the low frequency components of the input signal. Moreover,  we don't need the high frequency components to be phase compensated. So, first the signal $u(t)$ is filtered by a low pass filter, obtaining the signal $u_{l}(t)$. Next, the high frequency components are obtained by doing $u(t)-u_{l}(t)=u_{h}(t)$. $u_{l}(t)$ is limited by the first rate limiter and phase compensated. Then the output signal is added to $u_{h}(t)$ and fed into a second rate limiter. At this point only the component $u_{h}(t)$ is limited (and uncompensated). Note that both rate limiter have the same settings and are software rate limiters usually ~\cite[p. 3945]{paper3}. Compared to the previous filter, this one has better phase compensation, because high frequency components do not interfere with the phase compensation of low frequency components. The parameters for this circuit are tuned based on the bandwidth of the system and of the input signal. Moreover, because of the low pass filter, the output of the compensated rate limiter has lower amplitude compared to a normal rate limiter. Check figure \ref{fig:filter} for a scheme of the filter. 
\end{enumerate}	
Stability properties in closed loop depend on the presence of high frequency disturbances. If not presents, the closed loop system with compensated rate limiters  has better stability properties. With disturbances any of the filters can be the better one or the worst one, it depends on the rest of the loop ~\cite[p. 3946]{paper3}. Moreover, stability properties can always be predicted by using the Describing Function method ~\cite[p. 3946]{paper3}.