\section*{Problem}
\label{sec:prob}
The problem consists of the analysis of the pitch angle dynamics of a JAS 39 Gripen, a military aircraft. Specifically, Pilot Induced Oscillations are analysed. This kind of oscillations happen when the pilot panics and tries to give maximum command. The control system of the aircraft is based on gain scheduling  of speed and altitude in order to compensate for the non linearities of the aircraft. The case considered is for an aircraft of normal load $M=0.6$, altitude $1$km and speed below the speed of sound. The velocity is important since at higher velocities the aircraft is stable, whilst for lower is unstable. The direction in which the aircraft is pointing, with reference to the earth frame, define the pitch angle $\theta$, and the angle between the aircraft and its center of mass velocity is the angle of attack $\alpha$.\\ The aircraft has two controllable surfaces to control $\theta$: the elevator (installed on the tail of the aircraft), and the spoiler (installed on the wings of the aircraft). Their angles are denoted by $\delta_e, \delta_s$. The dynamics of the rudder servos are denoted by $x_e, x_s$ and the control inputs are $u_e$ the elevator command, and $u_s$, the spoiler command. We consider a linear model $\dot{x}=Ax+Bu$ where $x=(\alpha, \dot{\theta}, \theta, \delta_e, \delta_s, x_e,x_s)^T, u=(u_e,u_s)^T$. Notice that $B$ is made such that the control can affect only $x_e, x_s$.\\ The aircraft model is shown in figure \ref{fig:aircraft}.
The aircraft is controlled as follows:
$$u(t)=-Lx(t)+(K_f,K_f)^T u_{\text{pilot}}^f(t)$$
where $L$ is derived using linear quadratic control theory and $K_f$ is chosen such that the steady state gain is correct.
The internal loop, based on state feedback, stabilizes $\alpha, \dot{\theta}$. The outer loop denotes the pilot, where he control the pitch angle. The pilot signal is also filtered by a low pass filter, with time constant $T_f$. 
The analysis of the problem is broken down into the following six sub-problems:
\begin{enumerate}
\item{First the dynamical modes of the aircraft are analysed, and classified into flight dynamics mode or rudder dynamics mode. The model is then checked to see if it is stable or not.}
\item {A first nominal design for the state feedback, called \emph{design1}, is used. The $L$ matrix  and the eigenvalues of $A-BL$ analysed. Then we discuss why the pilot can be modelled by a PD-Controller with time delay. The system is then simulated, and the rudder angles analysed.}
\item{Now the Pilot Induced Oscillations (PIO) are analysed. The pilot is panicking and tries to compensate the error $\theta_{\text{ref}}-\theta$ with maximal command signals. The PD-Controller is replaced by a relay. The relay is analysed and the Describing Function Method is used to predict a possible limit cycle.} 
\item{Various designs, \emph{design2} and \emph{design3} are compared. The results are discussed and analysed using the Describing Function Method.}
\item{A control strategy that outperform the previous design to reduce the PIO amplitude is given. The performance is analysed and compared with the previous design}.
\item{(Extra) Rate limitations on the hydraulic servos are now included on the model. Rate limitations are discussed and the PIO are analysed based on the new model. Next a non-linear filter \cite{paper3}, that compensate the rate limiter, is analysed.}
\end{enumerate}